\documentclass[a4paper,12pt]{article}
\usepackage[top=2cm, bottom=2cm, left=2.5cm, right=2.5cm]{geometry}

\begin{document}
	Uma aplicação clássica da teoria de detecção é na área de radar. O enunciado dessa questão pode ser interpretado, para fins de contextualização, como um problema de detecção de um alvo em sistemas de sonar\slash radar. Considera-se que $N$ é um ruído que corrompe a detecção de um alvo, $Y$.
	
	$H_0$ é denominada de \textit{null hypothesis} e indica a hipótese na qual não há a presença do alvo. $H_1$, por outro lado, é denominado de \textit{alternative hypotesis}, e indica a hipótese da precença do alvo.
\end{document}